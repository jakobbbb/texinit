\documentclass[a4paper]{article}

% symlink to ~/.texinit/default.tex:
%   ln -s /home/jakob/uni/template.tex /home/jakob/.texinit/default.tex

%-----------------------------------------------------------------------
% LaTeX packages
%-----------------------------------------------------------------------

% Set language to German
\usepackage[ngerman]{babel}
\selectlanguage{ngerman}

\usepackage{textcomp}
\usepackage{amsfonts}
\usepackage{amsmath}
\usepackage{amsthm}
\usepackage{amssymb}
\usepackage{booktabs}
\usepackage[hmargin=1in,vmargin=1in]{geometry}
\usepackage[pdfborderstyle={/S/U/W 1}]{hyperref}
\usepackage{xspace}
\usepackage{lipsum}
\usepackage{graphicx}
\usepackage{float}
\usepackage{subfig}
%\usepackage[utf8]{inputenc}

\usepackage[symbol]{footmisc}
%\usepackage{indentfirst}
%\setlength\parindent{0pt} % disable indenting

% German bibliography
\usepackage{bibgerm}
\usepackage[square,numbers]{natbib}

\usepackage{array} % allows for things like `>{\bf}c` in tables

\usepackage{fancyhdr}

\pagestyle{fancy}
\fancyhf{}
\lhead{\textsc{{{DESTINATION}}}}
\rhead{\textsc{Donald Duck}}
%\rhead{\textsc{Goose, Duck, Tuna}}
%\cfoot{--~\thepage~--}
\fancyfoot[C]{--~\thepage~--}

\fancypagestyle{plain}{%
  \fancyhf{}
  \renewcommand*{\headrulewidth}{0pt}
  \fancyfoot[C]{--~\thepage~--}
}

\usepackage{titling}

% Footnotes w/ symbols, not numbers
%\renewcommand{\thefootnote}{\fnsymbol{footnote}}

% Fix `Package Fancyhdr Warning: \headheight is too small (12.0pt): Make it at least 12.37082pt`
%\setlength{\headheight}{12.38pt}

%-----------------------------------------------------------------------
% Theorems
%-----------------------------------------------------------------------

\newtheorem{thm}{Theorem}[section]
\newtheorem{cor}[thm]{Corollary}
\newtheorem{lem}[thm]{Lemma}
\newtheorem{prop}[thm]{Proposition}
\theoremstyle{definition}
\newtheorem{defn}[thm]{Definition}
\theoremstyle{remark}
\newtheorem{rem}[thm]{Remark}

%-----------------------------------------------------------------------
% Macros
%-----------------------------------------------------------------------

\newcommand{\pmat}[1]{\begin{pmatrix} #1 \end{pmatrix}}

%-----------------------------------------------------------------------

\begin{document}

%-----------------------------------------------------------------------
% Title
%-----------------------------------------------------------------------

\title{\textbf{%
    {{DESTINATION}}
}}

%\author{Untitled Goose, Donald Duck und Canned Tuna\\
%  \normalsize Matrikelnummern: 123456, 123456 und 123456\\
\author{Donald Duck\\
  \normalsize \href{mailto:donald.duck@example.org}
    {\nolinkurl{donald.duck@example.org}} \\
  \normalsize Matrikelnummer: 123456\\
  \vspace{3mm}
  \normalsize Bauhaus-Universit\"at Weimar\\
}

%\author{
%  Untitled Goose \\
%  \normalsize
%  Matr.-Nr.: 123456 \\
%  \normalsize
%  \href{mailto:untitled.goose@example.org}
%  {\nolinkurl{untitled.goose@example.org}}
%\and
%  Donald Duck \\
%  \normalsize
%  Matr.-Nr.: 123456 \\
%  \normalsize
%  \href{mailto:donald.duck@example.org}
%  {\nolinkurl{donald.duck@example.org}}
%\vspace{3mm}
%\and
%  Canned Tuna \\
%  \normalsize
%  Matr.-Nr.: 123456 \\
%  \normalsize
%  \href{mailto:canned.tuna@example.org}
%  {\nolinkurl{canned.tuna@example.org}}
%\vspace{5mm}
%\\
%Bauhaus-Universit\"at Weimar
%}

\maketitle

%-----------------------------------------------------------------------
% Content
%-----------------------------------------------------------------------

%-----------------------------------------------------------------------
%\vfill
%\noindent
%\rule{\textwidth}{0.4pt}
%
%\bibliographystyle{natdin}
%\bibliography{refs} % Entries are in the "refs.bib" file
%-----------------------------------------------------------------------
\end{document}
